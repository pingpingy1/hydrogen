\documentclass{article}

\usepackage{braket}
\usepackage{amsmath}
\usepackage{amssymb}
\usepackage{bbm}
\usepackage[margin=0.8in]{geometry}

\newcommand{\hilbert}{\mathcal{H}}
\newcommand{\abs}[1]{\left| #1 \right|}
\DeclareMathOperator{\spn}{span}
\renewcommand{\mathbb}{\mathbbm}
\newcommand{\comm}[2]{\left[ #1, #2 \right]}
\newcommand{\unitvec}[1]{\hat{\mathbb{#1}}}



\title{Hyperfocused Derivation of Hydrogen Energy Levels: From Scratch}
\author{Heewon Lee}
\date{\today}

\begin{document}

\maketitle
\section{State Representation}
\subsection{The Hilbert Space}
The defining postulate of quantum mechanics is that the state of a system
is described with an element in a Hilbert space.
Say the state of a system is a ket vector $\Ket{\alpha} \in \hilbert$.
An observable property is represented by an operator $A: \hilbert \rightarrow \hilbert$.
We presume the operator is normal, in which case it has a spectral decomposition
\[
    A = \int d\mu(\lambda) \lambda \Ket{\lambda}\Bra{\lambda}.
\]
A possible observed value is an eigenvalue $\lambda$ of the operator,
so all eigenvalues being real is equivalent to $A$ being Hermitian, that is $A^\dag = A$.
When the system $\Ket{\alpha}$ is observed with respect to the observable $A$,
it randomly collapses to one of its eigenkets $\Ket{\lambda}$
with probability $\abs{\Braket{\lambda | \alpha}}^2$.
The total probability adding up to 1 can be stated as
\[
    1
    = \int d\mu(\lambda) \abs{\Braket{\lambda | \alpha}}^2
    = \Bra{\alpha} \left( \int d\mu(\lambda) \Ket{\lambda}\Bra{\lambda} \right) \Ket{\alpha}
    = \Braket{\alpha | \alpha}.
\]
This is succinctly referred to as $\Ket{\alpha}$ being normalized.

\subsection{Time Evolution}
Suppose a system in state $\Ket{\alpha, t_0}$ at time $t_0$
evolves to some other state $\Ket{\alpha, t_0; t}$ at time $t$.
We make four assumptions:
\begin{enumerate}
    \item \emph{Time evolution is linear:}
        $\Ket{\alpha, t_0; t} = U(t_0, t)\Ket{\alpha, t_0}$
        for some linear operator $U(t_0, t): \hilbert \rightarrow \hilbert$.

    \item \emph{Probability is conserved:}
        $\Braket{\alpha, t_0; t | \alpha, t_0; t} = 1$.
        This is equivalent to $U^\dag(t_0, t) = U^{-1}(t_0, t)$,
        which we call unitary.

    \item \emph{Time evolution composes:}
        $U(t_1, t_2) U(t_0, t_1) = U(t_0, t_2)$ for $t_0 < t_1 < t_2$.

    \item \emph{Time evolution is continuous:}
        $\lim_{t \rightarrow t_0^+} U(t_0, t) = 1$.
\end{enumerate}
If we apply these to the infinitesimal evolution $U(t_0, t_0 + dt)$,
we find that there is some Hermitian operator $H(t)$ such that
\[
    U(t_0, t_0 + dt) = 1 - \frac{iH(t_0)dt}{\hbar}.
\]
This operator $H$ is the generator of time evolution,
which is also a description of the Hamiltonian/energy of a classical system.
Therefore, we extend this terminology and say $H(t)$ is the operator
corresponding to the Hamiltonian of the system.

\subsection{3D Space}
We want to describe the Hilbert space
\[
    \hilbert = \spn\Set{\Ket{\mathbb{r}'} | \mathbb{r}' \in \mathbb{R}^3}.
\]
An element of this space would take the form
\[
    \Ket{\alpha}
    = \int d^3x' \Ket{\mathbb{r}'} \Braket{\mathbb{r}' | \alpha}
    =: \int d^3x' \Ket{\mathbb{r}'} \psi_\alpha \left( \mathbb{r}' \right)
\]
where we have defined the wave function $\psi_\alpha\left(\mathbb{r}'\right) := \Braket{\mathbb{r}' | \alpha}$.
This definition leads to the delta normalization:
\[
    \Braket{\mathbb{r}'' | \mathbb{r}'} = \delta^3 \left( \mathbb{r}'' - \mathbb{r}' \right).
\]
A translation by a vector $\mathbb{a}$ would be represented with an operator $\mathcal{T}(\mathbb{a})$
such that $\mathcal{T}(\mathbb{a}) \Ket{\mathbb{r}'} = \Ket{\mathbb{r}' + \mathbb{a}}$.
From $\mathcal{T}(\mathbb{a}) = \int d^3x' \Ket{\mathbb{r}' + \mathbb{a}} \Bra{\mathbb{r}'}$,
we can prove compositionality $\mathcal{T}(\mathbb{b}) \mathcal{T}(\mathbb{a}) = \mathcal{T}(\mathbb{a} + \mathbb{b})$
and inversion $\mathcal{T}^\dag(\mathbb{a}) = \mathcal{T}(-\mathbb{a}) = \mathcal{T}^{-1} (\mathbb{a})$.
Assuming continuity, similar arguments as the time evolution operator allows us to write
\[
    \mathcal{T}(d\mathbb{r}') = 1 - \frac{i\mathbb{p} \cdot d\mathbb{r}'}{\hbar}
\]
for some Hermitian operator $\mathbb{p}$.
The two equations
\[
    \Braket{\mathbb{r}' | \mathcal{T}(d\mathbb{r}') | \alpha}
    = \Braket{\mathbb{r}' - d\mathbb{r}' | \alpha}
    = \Braket{\mathbb{r}' | \alpha} - d\mathbb{r}' \cdot \nabla' \Braket{\mathbb{r}' | \alpha}
\]
and
\[
    \Braket{\mathbb{r}' | \mathcal{T}(d\mathbb{r}') | \alpha}
    = \Braket{\mathbb{r}' | \alpha} - \frac{i}{\hbar} d\mathbb{r}' \cdot \Braket{\mathbb{r}' | \mathbb{p} | \alpha}
\]
can be compared to find
\[
    \Braket{\mathbb{r}' | \mathbb{p} | \alpha}
    = -i\hbar \nabla' \Braket{\mathbb{r}' | \alpha}.
\]
This representation allows for easy computation of the commutation relations:
\[
    \comm{x_i}{x_j} = \comm{p_i}{p_j} = 0,\quad
    \comm{x_i}{p_j} = i\hbar \delta_{ij}.
\]
Here, we have defined the position operators
\[
    \mathbb{r} := \int d^3x' \mathbb{r}' \Ket{\mathbb{r}'}\Bra{\mathbb{r}'}.
\]

\subsection{Rotation}
For a 3D rotation $R \in SO(3)$, we expect there to be a corresponding rotation operator
\[
    \mathcal{D}(R) \Ket{\mathbb{r}'} = \Ket{R\mathbb{r}'}.
\]
The rotation around a unit vector $\unitvec{n}$ by an angle $\phi$ can be found as
\[
    R\left(\phi\unitvec{n}\right) \mathbb{r}'
    = \cos\phi\mathbb{r}'
      + \left( 1 - \cos\phi \right) \left( \mathbb{r}'\cdot\unitvec{n} \right) \unitvec{n}
      + \sin\phi \unitvec{n}\times\mathbb{r}',
\]
which for small angles $\epsilon$, becomes
$R\left(\epsilon\unitvec{n}\right)\mathbb{r}' = \mathbb{r}' + \epsilon \unitvec{n}\times\mathbb{r}'$.
From this, we may write
\[
    \mathcal{D}\left(R\left(\epsilon\unitvec{n}\right)\right) \Ket{\mathbb{r}'}
    = \Ket{\mathbb{r}' + \epsilon\unitvec{n}\times\mathbb{r}'}
    = \mathcal{T}\left(\epsilon\unitvec{n}\times\mathbb{r}'\right) \Ket{\mathbb{r}'}
    = \left( 1 - \frac{i\epsilon}{\hbar} \mathbb{p}\cdot\left(\unitvec{n}\times\mathbb{r}'\right) \right) \Ket{\mathbb{r}'}.
\]
Hence, defining the angular momentum operator $\mathbb{L} := \mathbb{r}\times\mathbb{p}$,
we obtain $\mathcal{D}\left(R\left(\epsilon\unitvec{n}\right)\right) = 1 - \frac{i\mathbb{L}\cdot\epsilon\unitvec{n}}{\hbar}$.
The definition allows the following commutation relations:
\[
    \comm{L_i}{x_j} = \varepsilon_{ijk}x_k,\quad
    \comm{L_i}{r^2} = 0,\quad
    \comm{L_i}{p_j} = \varepsilon_{ijk}p_k,\quad
    \comm{L_i}{p^2} = 0,\quad
    \comm{L_i}{L_j} = \varepsilon_{ijk}L_k,\quad
    \comm{L_i}{\mathbb{L}^2} = 0.
\]
Notice that $\mathcal{D}(R_2)\mathcal{D}(R_1) = \mathcal{D}(R_2 R_1)$
and $\mathcal{D}^\dag(R) = \mathcal{D}\left(R^{-1}\right) = \mathcal{D}^{-1}(R)$,
implying $\mathbb{L}^\dag = \mathbb{L}$.
We can obtain the representation of $\mathbb{L}$ in spherical coordinates by comparing
\[
    \Braket{\mathbb{r}' | \mathcal{D}\left(\epsilon\unitvec{n}\right) | \alpha}
    = \Braket{\mathbb{r}' - \epsilon\unitvec{n}\times\mathbb{r}' | \alpha}
    = \Braket{\mathbb{r}' | \alpha} - \epsilon \left(\unitvec{n}\times\mathbb{r}'\right) \cdot \nabla' \Braket{\mathbb{r}' | \alpha}
\]
and
\[
    \Braket{\mathbb{r}' | \mathcal{D}\left(\epsilon\unitvec{n}\right) | \alpha}
    = \Braket{\mathbb{r}' | \alpha} - \frac{i\epsilon\unitvec{n}}{\hbar} \cdot \Braket{\mathbb{r}' | \mathbb{L} | \alpha}, 
\]
which then yields
\[
    \Braket{\mathbb{r}' | \mathbb{L} | \alpha}
    = -i\hbar\mathbb{r}' \times \nabla' \Braket{\mathbb{r}' | \alpha}
    = \left(
            - i\hbar \frac{\partial}{\partial\theta'} \unitvec{\phi}'
            + i\hbar \csc\theta' \frac{\partial}{\partial\phi'} \unitvec{\theta}'
      \right) \Braket{\mathbb{r}' | \alpha}
\]
or
\[
    \Braket{\mathbb{r}' | L_x | \alpha}
    = \left(
            -i\hbar \sin\phi' \frac{\partial}{\partial\theta'}
            -i\hbar \cot\theta' \cos\phi' \frac{\partial}{\partial\phi'}
      \right) \Braket{\mathbb{r}' | \alpha}
\]
\[
    \Braket{\mathbb{r}' | L_y | \alpha}
    = \left(
             i\hbar \cos\phi' \frac{\partial}{\partial\theta'}
            -i\hbar \cot\theta' \sin\phi' \frac{\partial}{\partial\phi'}
      \right) \Braket{\mathbb{r}' | \alpha}
\]
\[
    \Braket{\mathbb{r}' | L_z | \alpha}
    = -i\hbar \frac{\partial}{\partial\phi'} \Braket{\mathbb{r}' | \alpha}.
\]
We could square these and add them up to find the representation of $\mathbb{L}^2$,
but we can more easily get it by observing
\[
    \mathbb{L}^2
    = \varepsilon_{ijk}\varepsilon_{lmk} x_i p_j x_l p_m
    = \cdots
    = r^2 p^2 - \left(\mathbb{r} \cdot \mathbb{p}\right)^2 + i\hbar \mathbb{r} \cdot \mathbb{p},
\]
which thus leads to
\[
    \Braket{\mathbb{r}' | \mathbb{L}^2 | \alpha}
    = -\hbar^2 r'^2 \left(
            \nabla'^2 - \frac{\partial^2}{\partial r'^2} - \frac{2}{r'}\frac{\partial}{\partial r'}
      \right) \Braket{\mathbb{r}' | \alpha}
    = -\hbar^2 \left(
            \frac{1}{\sin\theta'} \frac{\partial}{\partial\theta'} \left(
                \sin\theta' \frac{\partial}{\partial\theta'}
            \right)
            + \frac{1}{\sin^2\theta'} \frac{\partial^2}{\partial\phi'^2}
      \right) \Braket{\mathbb{r}' | \alpha}.
\]

% =============================================
\section{Hydrogen Atom}
\subsection{Compatible Observables}
Suppose two observables $A$ and $B$ commute, i.e., $\comm{A}{B} = 0$.
We prove that the two have a common eigenbasis of $\hilbert$.
Let $\Ket{a, \lambda}$ denote an eigenbasis of $A$ with eigenvalue $a$.
Notice that
\[
    A\left(B\Ket{a, \lambda}\right) = BA\Ket{a, \lambda} = a\left(B\Ket{a,\lambda}\right),
\]
so $B\Ket{a,\lambda}$ is also an eigenket of $A$ with eigenvalue $a$.
This allows us to write
\[
    B\Ket{a, \lambda}
    = \int d\mu(\lambda') K(\lambda', \lambda) \Ket{a, \lambda'}
\]
where the kernel $K$ is defined to be
\[
    K(\lambda', \lambda)
    := \Braket{a, \lambda' | B | a, \lambda}.
\]
Assuming compactness of this kernel, therefore,
we can find another basis $\Set{\Ket{a, \rho}}$ for this eigenspace, where
\[
    \Ket{a, \rho} = \int d\mu(\lambda) L(\lambda, \rho) \Ket{a, \lambda}
\]
and
\[
    A\Ket{a, \rho} = a\Ket{a, \rho},\quad B\Ket{a, \rho} = b(\rho) \Ket{a, \rho}.
\]
Therefore, $A$ and $B$ are simultaneously diagonalizable.

\subsection{Angular Momentum Basis}
The Hamiltonian of a hydrogen atom is given by
\[
    H = \frac{p^2}{2\mu} - \frac{k}{r}\quad
    \left( k = \frac{e^2}{4\pi\epsilon_0} \right),
\]
which incidentally commutes with $\mathbb{L}^2$ and $L_z$
because the Coulomb potential shows spherical symmetry.
This allows us to declare a simultaneous eigenbasis $\Set{\Ket{Eab}}$
with eigenvalues $E, a, b$ for $H, \mathbb{L}^2, L_z$, respectively.

Let $L_\pm := L_x \pm iL_y$.
\[
    \comm{L_z}{L_\pm} = \pm \hbar L_\pm,\quad
    \comm{L_+}{L_-} = 2\hbar L_z,\quad
    L_+ L_- = L_x^2 + L_y^2 + \hbar L_z
\]
\[
    \Rightarrow \mathbb{L}^2
    = L_+ L_- + L_z^2 - \hbar L_z
    = L_- L_+ + L_z^2 + \hbar L_z
\]
We now try to find what eigenvalues $a, b$ are possible.
\[
    a
    = \Braket{Eab | \mathbb{L}^2 | Eab}
    = \left\lVert L_- \Ket{Eab} \right\rVert^2 + b^2 - \hbar b
    = \left\lVert L_+ \Ket{Eab} \right\rVert^2 + b^2 + \hbar b
    \Rightarrow a \geq b^2 + \hbar\abs{b}
\]
Hence, the range of $b$ is bounded for a given $a$, say between $b_{min}$ and $b_{max}$.
But notice that $L_z \left( L_\pm \Ket{Eab} \right) = (b \pm \hbar) \left( L_\pm \Ket{Eab} \right)$.
Thus, we must have $L_+ \Ket{Eab_{max}} = L_- \Ket{Eab_{min}} = 0$,
and also $b_{max} - b_{min} \in \hbar \mathbb{N}$.
This also implies
\[
    a = b_{max}^2 - \hbar b_{max} = b_{min}^2 + \hbar b_{min}
\]
\[
    \Rightarrow b_{max} = -b_{min} = \hbar j \; \left( j \in \frac{1}{2} \mathbb{N} \right),\quad
    a = j (j + 1) \hbar^2,\quad
    b = -j, -j+1, \dots, j-1, j.
\]

Now, let us employ the position-basis representation of $\mathbb{L}$ to get further restrictions.
\[
    \Braket{\mathbb{r}' | L_z | Ejm}
    = -i\hbar \frac{\partial}{\partial\phi'} \Braket{\mathbb{r}' | Ejm}
    = m\hbar \Braket{\mathbb{r}' | Ejm} \;
    \Rightarrow \Braket{\mathbb{r}' | Ejm} \propto e^{im\phi'}
\]
However, the wave function must be single-valued for $\phi'$ and $\phi' + 2\pi$;
hence, $m$ must be an integer and thus $j = l \in \mathbb{N}$.
\[
    \Braket{\mathbb{r}' | \mathbb{L}^2 | Elm}
    = -\hbar^2 \left(
            \frac{1}{\sin\theta'}\frac{\partial}{\partial\theta'}\left(
                \sin\theta'\frac{\partial}{\partial\theta'}
            \right)
            + \frac{1}{\sin^2\theta'}\frac{\partial^2}{\partial\phi'^2}
      \right) \Braket{\mathbb{r}' | Elm}
    = \hbar^2 l(l+1) \Braket{\mathbb{r}' | Elm}
\]
\[
    \Rightarrow \left(
        \frac{1}{\sin\theta'}\frac{\partial}{\partial\theta'}\left(
            \sin\theta'\frac{\partial}{\partial\theta'}
        \right)
        - \frac{m^2}{\sin^2\theta'} - l(l+1)
    \right) \Braket{\mathbb{r}' | Elm} = 0
\]
Thus, we may decompose the wavefunction into the radial part and the angular part:
\[
    \Braket{\mathbb{r}' | Elm} = R(r') \Theta_{lm}(\theta') e^{im\phi'}.
\]

\subsection{Energy Levels}
We find the bound states of the hydrogen atom, i.e., $E < 0$.
\[
    \Braket{\mathbb{r}' | H | Elm}
    = \left( -\frac{\hbar^2}{2\mu}\nabla'^2 - \frac{k}{r'} \right) \Braket{\mathbb{r}' | Elm}
    = \left( -\frac{\hbar^2}{2\mu} \left(
                \frac{\partial^2}{\partial r'^2}
                + \frac{2}{r'}\frac{\partial}{\partial r'}
                - \frac{l(l+1)}{r'^2}
            \right) - \frac{k}{r'}
      \right) \Braket{\mathbb{r}' | Elm}
\]
\[
    \Rightarrow \frac{d^2 R}{dr'^2}
    + \frac{2}{r'}\frac{dR}{dr'}
    + \left( \frac{2\mu E}{\hbar^2} - \frac{l(l+1)}{r'^2} + \frac{2\mu k}{\hbar^2 r'} \right) R
    = 0
\]
Let us define $\kappa := \frac{\sqrt{2\mu\abs{E}}}{\hbar}$ and $\lambda := \frac{\mu k}{\hbar}$.
For $r'\rightarrow\infty$, $\frac{d^2 R}{dr'^2} \approx \kappa^2 R \Rightarrow R \sim e^{\pm \kappa r'}$.
For $r'\rightarrow0+$, similarly, we have
\[
    \frac{d^2 R}{dr'^2} + \frac{2}{r'}\frac{dR}{dr'} - \frac{l(l+1)}{r'^2} \approx 0
    \Rightarrow R \sim r'^l \text{ or } r'^{-l-1}.
\]
For normalizability, hence, we require that $R(r') = f(r') r'^l e^{-\kappa r'}$
for some ``well-behaved'' function $f(r')$.
Substitution yields the new differential equation
\[
    r'\frac{d^2 f}{dr'^2} + 2(l + 1 - \kappa r')\frac{df}{dr'} + 2(\lambda - (l+1)\kappa)f = 0.
\]
Suppose $f(r') = \sum_{k=0}^\infty c_k r'^k$ for unknown coefficients $c_k$.
The differential equation then becomes
\[
    \sum_{k=0}^\infty \left(
        (k+1)(k+2l+2)c_{k+1} - 2(\kappa(k + l + 1) - \lambda)c_k
    \right) r'^k = 0
\]
\[
    \Rightarrow c_{k+1} = \frac{2(\kappa(k+l+1) - \lambda)}{(k+1)(k+2l+2)} c_k.
\]
Now, if every coefficient were nonzero, then for large $k$,
\[
    c_k \sim \frac{(2\kappa)^k}{k!}
    \Rightarrow f(r') \sim e^{2\kappa r'},
\]
which would render $R(r')$ non-normalizable.
Therefore, the sequence $\Set{c_k}$ must terminate, i.e.,
there must exist some $k_0 \in \mathbb{N}$ such that $\kappa(k_0+l+1) - \lambda = 0$.
Relabelling as $n := k_0 + l + 1 \in \mathbb{Z}_{>0}$ (rendering $l = 0, 1, \dots, n - 1$),
we finally obtain
\[
    E_n = -\frac{\mu k^2}{2\hbar^2 n^2} = -\frac{Z^2 e^4 \mu}{32\pi^2 \epsilon_0^2 \hbar^2 n^2},
\]
or using the fine-structure constant $\alpha := \frac{e^2}{4\pi \epsilon_0 \hbar c}$,
\[
    E_n = -\frac{\mu c^2 Z^2 \alpha^2}{2n^2} \approx -(13.6 \mathtt{ eV}) \frac{Z^2}{n^2}
\]
thus concluding the derivation.
\hfill$\square$
\end{document}

